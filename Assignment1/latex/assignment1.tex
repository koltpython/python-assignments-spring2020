\documentclass[a4paper]{article}

\usepackage[english]{babel}
\usepackage[utf8]{inputenc}
\usepackage{amsmath}
\usepackage{graphicx}
\usepackage[colorinlistoftodos]{todonotes}
\usepackage[useregional]{datetime2}
\usepackage{fancyhdr}
\usepackage{titlesec}
\usepackage{listings} 
\usepackage[hidelinks]{hyperref}
\usepackage{caption}
\usepackage{titling}

\usepackage{color}

 \usepackage{wrapfig}

\setcounter{secnumdepth}{4}

\setlength{\droptitle}{-4em}

\setlength{\intextsep}{6pt plus 2pt minus 2pt}


%Project Title
\newcommand{\projectTitle}{Tic-Tac-Toe}
%Homework Number
\newcommand{\assignmentNumber}{1}

%Submission date goes here.
%Format: #Day of Month #Hour:#Minute AM/PM
\newcommand{\projectDate}{3rd of March 11:59 PM}

%Contact to reach.
\newcommand{\contactName}{Hasan Can Aslan}
\newcommand{\contactMail}{haslan16@ku.edu.tr}

%Define colors as shown below to use in text.
\definecolor{Red}{RGB}{255, 0, 0}
\definecolor{Green}{RGB}{0, 255, 0}
\definecolor{Blue}{RGB}{0, 0, 255}

\author{Hasan Can Aslan}

\title{\projectTitle}

\date{Submission Date: \projectDate}

\begin{document}

\maketitle

\lstset{language=Python}
\pagestyle{fancy}
\fancyhf{}
\chead{\projectTitle}
\rhead{Assignment \#\assignmentNumber}
\lhead{KOLT Python}
\lfoot{\nouppercase{\leftmark}}
\rfoot{Page \thepage}
\thispagestyle{fancy}
\renewcommand{\headrulewidth}{0.4pt}
\renewcommand{\footrulewidth}{0.4pt}


\section{Introduction}

On this project, you will $\dots$

%Do not change this section.
\subsection{Submission}

\noindent Additional notes:
\begin{itemize}
\item Using the naming convention properly is important, failing to do so may be \textbf{penalized}.
\item \textbf{Do not} use Turkish characters when naming files or folders.
\item Submissions with unidentifiable names will be \textbf{disregarded} completely. (ex. "homework1", "project" etc.) 
\item Please write your name into the Java source file where it is asked for.
\end{itemize}

%Do not change this section.
\subsection{Academic Honesty}
Koç University's \emph{\href{https://vpaa.ku.edu.tr/sites/vpaa.ku.edu.tr/files/Misc_Documents/Statement_on_Academic_Honesty.pdf}{Statement on Academic Honesty}} holds for all the homeworks given in this course. Failing to comply with the statement will be penalized accordingly. If you are unsure whether your action violates the code of conduct, please consult with your instructor.

\subsection{Aim of the Project}
In your homework assignments, the \textbf{functionality} and \textbf{style} of your programs are \textbf{both important.}  A program that “works” is not necessarily a good program. A good program is \textbf{comprehensible, readable and well structured.} Therefore you’re expected to do \textbf{“Stepwise Refinement”} to decompose the main problem into simpler subtasks and \underline{\textbf{implement helper methods}} for these subtasks.  You’re also expected to write comments if needed, be careful about indentation and use descriptive names for helper methods. Even if your program functions well according to the project requirements, you may not be able to get full credit if the style of your code has problems.

%Do not change this section.
\subsection{Given Code}
This part is \textbf{optional} but advised as it will allow you to understand the given partitions of the code better. \textbf{Do not} change anything in the code if it is indicated to you with a comment. The code given to you has something called \textbf{JavaDoc} comments above all the methods. These comments allow you to view various information about the method when you mouse over the name of the method. Below are the methods given to you in the code with their explanation.

%Explain the methods you have given in this section.
%Examples are given below.
\subsubsection{Given Methods}
\label{themeMethod}
%This is a list. Lists always begin with this tag. "itemize" describes the list, not begin.
\begin{itemize}

%This is a list item. There is no need to indicate the end for an item as it ends at the next item tag.
\item

%This is a code block. "lstlisting" describes a code block.
\begin{lstlisting}
void playThemeSong(String fileLocation)
\end{lstlisting}
Plays the original \textbf{"Overworld Theme"} of Nintendo's \textbf{Super Mario Bros.} game.
\item
\begin{lstlisting}
void playVictorySong(String fileLocation)
\end{lstlisting}
Plays the original \textbf{"Victory Theme"} of \textbf{Super Mario Bros.} game.

\end{itemize}

%Do not change this section.
\subsubsection{Given Constants}
Two constants, i.e THEME\_SONG and VICTORY\_SONG, are given at the bottom of the project. You will use this constants as \textbf{arguments} for given methods. Since you did not learn methods in detail yet, \textbf{we will provide necessary code in corresponding sections of the project.}

%Do not change this section.
\subsection{Further Questions}
For further questions \textbf{about the project} you may contact \textbf{\contactName} at \href{mailto:\contactMail}{\mbox{[\contactMail]}}. Note that it may take up to 24 hours before you receive a response so please ask your questions \textbf{before} it is too late. No questions will be answered when there is \textbf{less than two days} left for the submission.

%This allows you to start a new page regardless of where the previous page ends. Please try to separate sections properly, however refrain from leaving extensive amounts of blank space as this may cause the students to think that the project file ends there.
\newpage

\section{Given Worlds}
\label{worlds}
There are two worlds available to you in project to test your code. However, your solution \textbf{must be general enough to work on any world with given conditions.}

%This is a figure. image.png is the path of the image file that will be used.
% \begin{figure}[!htb]
% \centering
% \includegraphics[width=1\textwidth]{worlds.png}
% \caption{Worlds given at project.}\label{fig:worlds}
% \end{figure}

\noindent All of the worlds (two that you are given and \textbf{others that will be used for testing your code}) consist of \textbf{eight} parts. All of the parts conform to some specific rules. These rules are explained in detail in the following sections.

%Number of task groups and tasks are dependent on the project. Feel free to change accordingly.
\subsection{Collecting three coins from  \emph{\href{https://www.mariowiki.com/\%3F_Block}{"? Blocks"}} (15 pts.)}
In the beginning of the project, you are encouraged to play the theme song to make yourselves and Karel feel better.
\newline
\noindent \textbf{Note:} Playing the theme song is \textbf{neither mandatory nor graded.} However, you can use the given method (\ref{themeMethod}) to play the theme song as given below.

\begin{lstlisting} 
playThemeSong(THEME_SONG);
\end{lstlisting} 

\noindent \textbf{Important Note:} The second street (the second lowest level) is referred as \textbf{ground level} through-out the document.



% \begin{wrapfigure}{l}{0.5\textwidth} %this figure will be at the right
%     \centering
% \frame{\includegraphics[width=0.5\textwidth]{part1.png}}
% \caption{Examples of Part 1 maps.}
% \end{wrapfigure}
\textbf{Rules for Part 1}
\begin{enumerate}
\item Karel starts the leftmost corner of the ground level, facing East.
\item The corner at the intersection of the column of the first coin and ground level is painted WHITE.
\item Shape of the part with length five is always as shown in the images of Figure 2 (altering between bricks
and coins).
\end{enumerate}


\subsection{Pipes (15 pts.)}
After Karel passed the ? Blocks, he will encounter two pipes. Karel needs to climb to the first pipe and jump to the second pipe while collecting all the coins (or in his world beepers) between them.



% \begin{figure}[!htb]
% \centering
% \frame{\includegraphics[width=0.75\textwidth]{part2.png}}
% \caption{Pipes}\label{fig:part2}
% \end{figure}

\textbf{Rules for Part 2:}
\begin{enumerate}
\item Pipes can have any length.
\item Coins are arranged in a rectangular order.
\item There is at least one empty corner between the highest coin and the ceiling.
\\
\end{enumerate}


\subsection{Bricks (15 pts.)}
\noindent On this part, Karel needs to jump over two bricks to continue his adventure.

% \begin{figure}[!htb]
% \centering
% \includegraphics[width=0.75\textwidth]{part3.png}
% \caption{Bricks}\label{fig:part3}
% \end{figure}

\textbf{Rules for Part 3:}
\begin{enumerate}
\item Bricks can have any length, however, they are always one corner thick.
\item Karel knows that \textbf{second brick is always higher than the first brick.}
\item First brick always starts at \textbf{one corner east from lava's column}.
\item Lava is not covered with a wall. Karel can't directly dive into lava, however, he \textbf{can visit even lava's neighbor corners} without getting hurt. 
\\
\end{enumerate}


\subsection{Downstairs (15 pts.)}


% \begin{wrapfigure}{r}{0.5\textwidth} %this figure will be at the right
%     \centering
% \frame{\includegraphics[width=0.5\textwidth]{part4.png}}
% \caption{Examples of Downstairs}
% \end{wrapfigure}
On this part, Karel needs to climb to the top of stairs and then descend.
\newline
\newline
\textbf{Rules for Part 4}
\begin{enumerate}
\item Stairs can be at any height.
\item Stairs always descend by one at each step.
\end{enumerate}



\subsection{Coin On Top (5 pts.)}

After descending from stairs, Karel needs to climb to the pipe and collect the coin on top of it.

% \begin{wrapfigure}{r}{0.3\textwidth} %this figure will be at the right
%     \centering
% \frame{\includegraphics[width=0.3\textwidth]{part5.png}}\\
% \caption{An invalid combination of downstairs and pipe.}	
% \end{wrapfigure}


\textbf{Rules for Part 5}
\begin{enumerate}
\item The pipe can be at any height.
\item There is at least \textbf{two empty corners between downstairs and the pipe.}
\end{enumerate}

\subsection{Upstairs (15 pts.)}

% \vspace{3.5 em}
% \begin{figure}[!htb]
% \centering
% \frame{\includegraphics[width=0.75\textwidth]{part6.png}}
% \caption{Upstairs Examples}\label{fig:part6}
% \end{figure}

\textbf{Rules for Part 6}
\begin{enumerate}
\item Stairs can be at any height.
\item Stairs always ascend by one at each step.
\end{enumerate}

\newpage

\subsection{\emph{\href{https://www.mariowiki.com/Goal_Pole}{"Flagpole"}} (15 pts.)}
On this part, Karel needs to ascend to the flag,\\
 and then lower it step by step.

\textbf{Important:} You have to animate this by \textbf{picking/putting beepers and painting corners} as shown below.


% \begin{figure}[!htb]
% \centering
% \includegraphics[width=1\textwidth]{part7.png}
% \caption{Illustration of how Karel needs to descend the flag.}\label{fig:part7}
% \end{figure}

\textbf{Rules for Part 7}
\begin{enumerate}
\item Flagpole is always in between two bricks and it starts from one corner above bricks.
\item The flagpole can have any length.
\item There is at least one empty corner between top of of the flagpole and ceiling.
\end{enumerate}


\subsection{Walk to Victory! (5 pts.)}
Congratulations! You are just a few steps away from victory. 
Karel needs to find the castle's door and go there. After Karel arrived to door, i.e. finished the level, you must play the Victory Theme by using given method as given below:

\begin{lstlisting} 
playVictorySong(VICTORY_SONG);
\end{lstlisting} 


\textbf{Hint:} The door has the color \textbf{DARK\_GRAY}.



%Do not change this section.
\subsection{End of Project}
Your project ends here. You may continue to tinker with the code to implement any desired features and discuss them with your section leader. However, \textbf{do not} include any additional features that you implement after this point in to your submission.  
\\

\noindent \textbf{Final Warning: Do not include anything beyond this point to your submission. Points may be deducted from your grade as additions might alter the normal behavior of your code.} 

\end{document}